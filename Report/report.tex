\documentclass[a4paper,12pt,oneside]{article}
\usepackage[english]{babel} % für die deutsche Sprache
\usepackage[utf8]{inputenc} % Für die direkte Eingabe von Umlauten im Editor u.a.
\usepackage[T1]{fontenc}
\usepackage[dvipsnames]{xcolor}
\usepackage{listings}
\usepackage{fancyhdr} % Für Kopf- und Fußzeilen
\usepackage{graphicx} % Zum Laden von Graphiken
\usepackage{setspace} % Paket zum Setzen des Zeilenabstandes
\usepackage{helvet}
\usepackage{mathptmx}
\usepackage{float}
\usepackage{caption}
\usepackage{ragged2e}
\usepackage{gensymb}
\usepackage{amsmath}
\usepackage{makecell}
\usepackage{amsmath}
\usepackage[colorlinks,pdfpagelabels,pdfstartview=FitH,
bookmarksopen=true,bookmarksnumbered=true,linkcolor=black,
plainpages=false,hypertexnames=false,citecolor=black]{hyperref} % Für Verlinkungen

%%--------------
%\usepackage[normalem]{ulem} % Für das Unterstreichen von Text z.B. mit \uline{}
\usepackage[left=2cm,right=2cm,top=1.5cm,bottom=1cm,
textheight=245mm,textwidth=160mm,includeheadfoot,headsep=1cm,
footskip=1cm,headheight=14.599pt]{geometry} % Einrichtung der Seite 

\onehalfspacing % Zeilenabstand auf 1,5-zeilig setzen
\setlength{\parindent}{0pt} % Einrückung nach Zeilenabstand unterdrücken

  \fancypagestyle{TOC}{% Seitenlayout Inhaltsverzeichnich, Abbildungsverzeichnis
  \fancyhf{} % clear all header and footer fields
  \fancyfoot[R]{\thepage} 
  \renewcommand{\headrulewidth}{0pt}
  \renewcommand{\footrulewidth}{0pt}}

\lstset{
  basicstyle=\color{darkgray},
%  backgroundcolor=\color{darkgray},
  showstringspaces=false,
  commentstyle=\color{ForestGreen},
  breaklines=true,
  postbreak=\mbox{\textcolor{red}{$\hookrightarrow$}\space},
  columns=fullflexible,
  keywordstyle=\color{darkgray},
  language=bash,
  emph={\$},
  emphstyle={\color{blue}\bfseries}
}
  
\lstset{literate=%
    {Ö}{{\"O}}1
    {Ä}{{\"A}}1
    {Ü}{{\"U}}1
    {ß}{{\ss}}1
    {ü}{{\"u}}1
    {ä}{{\"a}}1
    {ö}{{\"o}}1
    {-}{{-}}1
    {-->}{{ $\Rightarrow$ }}1
    {>>}{{\glqq}}1
    {<<}{{\grqq}}1
}

\begin{document}

\pagestyle{empty}
\begin{titlepage}
	\includegraphics[scale=1.00]{sources/logo_TH-Koeln_CMYK_22pt-eps-converted-to.pdf}\\
	\begin{center}
		\large
		University of Applied Sciences Cologne\\
		Special Aspects of Mobile Autonomous Systems\\
		\includegraphics[scale=1.0]{sources/TH.PNG}\\
		\vspace{1cm}
		\textsc{Report}\\
		\vspace{2cm} % Vertikaler Abstand von 1cm erzeugen
		\LARGE
		Autonomous Object Hunting\\
		%\Large
		%Ggf. Untertitel\\
		\vspace{3cm}
		\large
		\vspace{1.0cm}
		by:\\
		\textsc{Tim Mennicken}\\
		\textsc{Manuel Audran}\\
		\textsc{Robert Rose}
		\vspace{1cm}

		\vspace{5cm}
		Cologne, \today
	\end{center}    
\end{titlepage}

\newpage

\thispagestyle{fancy}
\fancyhead{}
\fancyhf{}
\renewcommand{\headrulewidth}{0pt}
\renewcommand{\footrulewidth}{0.4pt}
\fancyfoot[L]{AMS Project - Abstract}
\fancyfoot[R] {}

\section*{Abstract}

The purpose of this project is to build a mobile robotic system that is able to move in an arbitrary environment while searching for an object. It avoids obstacles, draws a map of the already navigated way and "hunts" (searches and identifies) for an unspecified object. The exact shape of the target object is variable and can be specified in the source code. The hunt is over, if the object is identified with at least a pre-defined certainty.\\

For the purpose of object detection a camera module in conjunction with image processing in a neural net is used. The system has several sensors to monitor its own movement as well as orientation in space and to detect surroundings. Movement is archived through several driven wheels. The camera module is able to move in an angle of 360 degrees.\\

While the basic navigation is carried out on the mobile system itself, path mapping and object detection are outsourced to external hardware. This requires a reliable way to transfer data from and to the appended hardware. Any device that supports WiFi as well as TCP/\,IP is able to communicate with the system. As the system should work in any location without the need of fixed network infrastructure (e.g. a router), it creates a WiFi hotspot for external devices. The system offers a TCP socket on a defined port. If any device establishes a TCP connection, data can be exchanged via well defined data structures.\\

This work implements design and realization of the hardware as well as the software of the system. The software is designed object oriented. All used hardware components are instantiated through regarding software classes. Furthermore an inter-process communication between different systems is realized via TCP/\,IP connections. The system is working reliable and fulfills most of the tasks defined in the beginning of the project. Only the drawing of a map of the surroundings could not be implemented completely.\\

There are two minor problems, which could not be fairly addressed in this work. On the one hand the used distance sensors are not able to deliver reliable measurement results in all circumstances, on the other hand the synchronization between implemented threads and processes can be improved further. The impact of these inconveniences are overcome by a proper error handling, which admittedly not cures the root reason.

\newpage
\cleardoublepage
\pagestyle{TOC}
\pagenumbering{roman}
\tableofcontents

\newpage
\listoffigures

\newpage
\listoftables

\cleardoublepage
\pagestyle{fancy} % Kopf- und Fußzeilen aktivieren (=> Paket "fancyhdr")
\fancyhead{}
\fancyhf{}
\renewcommand{\headrulewidth}{0pt}
\renewcommand{\footrulewidth}{0.4pt}
\fancyfoot[L]{AMS Project - Report}
\fancyfoot[R] {Page \thepage}
\pagenumbering{arabic}

\section{Introduction}

This research project has the objective to design a mobile system that is able to search and identify an object while navigating in an unknown environment. Furthermore it builds a map of its surroundings and logs the navigated route. The action of searching and identifying an object will further be referenced as hunting. As the mobile system on its own only offers limited computing powers, it will outsource computing intensive tasks to external computers. 

\newpage

\section{Hardware}

The hardware of the device can be separated in two subsystems. On the one hand, the robot, which will navigate in the environment, on the other hand the external computer with a static location. This section only focuses on the hardware of the robot. The hardware of the external computer is not further described in this report as any arbitrary recent computer with a working WiFi interface can be used.

Following you will find a short description and the use case of each part used in the robot. All parts mentioned in this section will be combined to one mobile robotic system.

\subsection{Chassis}\label{subsec:chassis}

The chassis forms the foundation of the robot. It consists out of two identical floors, arranged above each other. 

\subsection{Gear Motors}\label{subsec:gear_motors}

Every movement the robot as a unit can make, is realized through four direct current gear motors, each attached to a wheel. Since the \nameref{subsec:processing_unit} can not operate the gear motors directly, dedicated motor drivers are needed in order to operate the gear motors properly. 

\subsection{Stepper Motor}\label{subsec:stepper_motor}

\subsection{Revolution Sensors}\label{subsec:revolution_sensor}

\subsection{Ultra Sonic Sensors}\label{subsec:ultra_sonic_sensor}

\subsection{Power Unit}\label{subsec:power_unit}

\subsection{Processing Unit}\label{subsec:processing_unit}

\subsection{PCB Board}\label{subsec:pcb_board}



\newpage

\section{Internal Software}

The in this section described processes are internal processes, i.e. running on the robot itself. In total there are three custom processes running on the robot. Figure \ref{fig:software} shows the structure of the developed software. This section focuses on the base and the navigation process. The image streaming process is referenced in section \ref{sec:image_processing}.

\begin{figure}[H]
\centering
\includegraphics[scale=0.6]{sources/software_structure.png}
\caption[Software structure]{Software structure}
\label{fig:software}
\end{figure}

\subsection{Design Principles}

The internal software underlies software specifications, which are defined at the beginning of the project. They define among other things how the system as a whole will behave and handle dedicated tasks, i.e. communication. The compliance of these principles ensure that the developed software works in the expected manner and offers the possibility for adding further processes in the future.

\begin{itemize}
\itemsep0em
\item Communication
	\begin{itemize}
	\item Via TCP/\,IP
	\item Independent from other devices
	\item Beyond system border
	\end{itemize}
	
\item Economical
	\begin{itemize}
	\item No unnecessary CPU consume
	\item Fast reaction times
	\item Event driven
	\end{itemize}
	
\item Documentation
	\begin{itemize}
	\item Complete
	\item Easy accessible
	\end{itemize}
\end{itemize}

\subsection{Base Process}

\subsection{Navigation Process}



\newpage

\section{Image Processing}\label{sec:image_processing}

\newpage

\section{Routing \& Mapping}

A further part of the vehicle is to log the driven route and build a map by detecting the environment.
As shown in figure \ref{route_map} the vehicle drives autonomous threw an unknown environment. During the
ride all objects besides the vehicle will be detected by sensor. The driven route will be logged by
storing the events of drive like orientation changes.\\
\begin{center}
\begin{minipage}{0.45\textwidth}
\label{route_map}
\includegraphics[page=3,scale=1]{sources/mapping/example_tunnel.pdf}
\captionof{figure}{Routing \& Mapping}
\end{minipage}
\end{center}

\subsection{Detection}

\subsubsection{Routing}

\begin{minipage}{0.5\textwidth}
By the installed hardware it is possible to measure the current orientation (angle) and the current speed. These values can be used to calculate the movements of the vehicle.\\
At the start of the processes and the vehicle an initialization will be proceeded, which sets the current position on an imaginary coordinate system at the point (the coordinates) (0,0)[x,y]. From this point the movements of the vehicle will be logged by calculating every new position respectively coordinate in a list.
\end{minipage}
\begin{minipage}{0.5\textwidth}
\centering
	\includegraphics[scale=0.6]{sources/mapping/initial_position.pdf}
	\captionof{figure}{Initial position}
\end{minipage}

\begin{minipage}{0.5\textwidth}
\centering
	\includegraphics[scale=0.6]{sources/mapping/orientation.pdf}
	\captionof{figure}{Vehicle orientation}
\end{minipage}
\begin{minipage}{0.5\textwidth}

By using the smart movement sensor the orientation respectively the angle of the vehicle can measured, in relation to the initial position / orientation which is 0\degree.

By using the smart movement sensor the orientation respectively the angle of the vehicle can measured, in relation to the initial position / orientation which is 0\degree.

\end{minipage}

\begin{minipage}{0.5\textwidth}
The vehicle provides an revolution sensor by which one the current speed $v$ of the vehicle can be measured. Every time the speed changes the cuttent time will be stored. With speed $v$ and the time slot $t$ the driven distance $s$ during this time slot can be calculated by $s=\frac{v}{t}$.
\end{minipage}
\begin{minipage}{0.5\textwidth}
	\centering
	\includegraphics[scale=0.8]{sources/mapping/distance.pdf}
	\captionof{figure}{Calculated driven distance}
\end{minipage}
\\
\\
By the resulting driven distance and the orientation of the vehicle the coordinates (related to the initial position and the following positions) can be calculated. The order of these coordinates results in the driven route.

\begin{minipage}{\textwidth}
	\centering
	\includegraphics[scale=0.6]{sources/mapping/orientation_distance.pdf}
	\includegraphics[scale=0.6]{sources/mapping/route.pdf}
	\captionof{figure}{Logged route information}
\end{minipage}
\\
\\
By the information of the driven route the current position of the vehicle is known.



\subsubsection{Mapping}

To detect object around the vehicle respectively the environment ultra sonic sensors are used. On the vehicle four sensors are installed, one at the front, rear, left and right.\\
The detected objects will be also logged by coordinates in relation to the current position and orientation of the vehicle.\\
In the beginning the sensor data will be read. These data will stored in coordinates which will adjust by adding the orientation of the vehicle to calculations. Finally, the calculated coordinates will be added to the current position coordinates of the vehicle to result the object position in relation to the initial point (0,0).

\begin{center}
	\includegraphics[page=1,scale=0.6]{sources/mapping/orientation_objectdistance.pdf}
	\includegraphics[page=2,scale=0.6]{sources/mapping/orientation_objectdistance.pdf}
	\includegraphics[page=3,scale=0.6]{sources/mapping/orientation_objectdistance.pdf}
	\captionof{figure}{Smart movement and ultra sonic sensor}
\end{center}

\newpage

\subsection{Software}

The routing \& mapping progress is a self-contained process as described in the beginning of this report and has been developed with Java.\\
\\
It contains several classes which will be show following:

\begin{table}[h]
\begin{center}
	\begin{tabular}{|p{0.3\textwidth}|p{0.7\textwidth}|}
		\hline
		Class & Description\\\hline\hline
		ProcessMain & Includes the main-methode to start the process by calling the method \textit{processControl()} of the class ProcessControl. It also may be used call additional method of further classes which may be added.\\\hline
		ProcessControl & Calls the relevant methods to run the important tasks.\\\hline
		SystemData & Stores the system relevant information like port or communication codes.\\\hline
		NetworkInterface & Establishes the network communication with the navigation process.\\\hline
		LogFile & Creates and edits the log files.\\\hline
		MessageAnalysis & Checks the received messages and their contents and handles the following processing\\\hline
		MapData & Stores and handles the data for the route and map building.\\\hline
		Calculations & Offers the methods for the calculations.\\\hline
		DrawRoute & Includes the methods for the route.png to be created.\\\hline
		DrawMap & Includes the methods for the map.png to be created.\\\hline
	\end{tabular}
	\caption{Process classes}
	\end{center}
\end{table}


As described before the relevant methods for this process will be called by the method \textit{processControl()}, which is shown below. To describe this the sequence more understandably the program sequence is graphically shown in figure \ref{programSequence}

\lstset{language=Java,
   basicstyle=\small,
   keywordstyle=\color{blue!80!black!100},
   identifierstyle=,
   commentstyle=\color{green!50!black!100},
   stringstyle=\ttfamily,
   breaklines=true,
   numbers=left,
   numberstyle=\small,
   frame=single,
   backgroundcolor=\color{blue!3}
}
\lstset{language=Java}

\begin{lstlisting}
public void processControl() {
	processInitialisation();
	while(!system.endOfProcess) {
		analysis.checkMessage(network.readSocket());
	}
	System.out.println("\nBuilding route...");
	drawRoute.buildMap(mapData.routeX, mapData.routeY, "Route", routeCoordinates);
	System.out.println("route.png created.\nBuilding map...");
	drawMap.buildMap(mapData.mapX, mapData.mapY, "Map", mapCoordinates);
	System.out.println("map.png created.\n");
		
	System.out.println("System ends.");
}
\end{lstlisting}
\captionof{lstlisting}{Method \textit{processControl()}}

In the beginning an initialization process will be proceeded. This includes the establishment of the network communication with the navigation process. Also, the first data of the sensor will be received to initialize the start position of the vehicle and respectively the center of the coordinate system (0,0).\\


\begin{minipage}{0.4\textwidth}
The code of the process includes a boolean variable which will be set true if the navigation process wants the process to be ended. This variable will check every time the while-loop start at the beginning. After this check the process wait for a message to be received in depend on the type of information to be received.\\
There are two options of informations to be received. The message could have the length of one byte, then this byte gives informations if the process has to stop, the vehicle rotates or which type of values will be send in the message. With that type of information the knows the byte array length of the next message. \\
If the message is longer than one byte the message includes sensor data and can be directly transfer to the methods which handle value data.\\
\end{minipage}
\begin{minipage}{0.6\textwidth}
\label{programSequence}
\begin{center}
	\includegraphics[scale=0.6]{sources/mapping/program_sequence.pdf}
	\captionof{figure}{Program sequence}
\end{center}
\end{minipage}

Finally the while-loop start at the beginning till the navigation process sends the termination sequence.\\
If this sequence will be received the while-loop stops and the methods to build the grphicals files for the route and the map will be called.

\begin{center}
	\includegraphics[scale=0.7]{sources/mapping/communication.pdf}
	\captionof{figure}{Principle commuication}
\end{center}


During the process run all the calculated coordinates of the route and the map respectively the objects will be stored in log files. At the end of the process the classes DrawRoute and DrawMap read the named files to get the coordinates. By these coordinates the wanted .png files will be created as shown below.

\begin{minipage}{0.5\textwidth}
\begin{center}
	\includegraphics[scale=0.015]{sources/mapping/Route.png}
	\captionof{figure}{Route.png}
\end{center}
\end{minipage}
\begin{minipage}{0.5\textwidth}
\begin{center}
	\includegraphics[scale=0.015]{sources/mapping/Map.png}
	\captionof{figure}{Map.png}
\end{center}
\end{minipage}


\subsection{Problems}

Unfortunately some problems with the ultra sonic sensors occurred.\\
The measured distance of the objects are not continuously correct respectively useful. By this error it is not possible to build an adequate map. The main object which are visible in the map build the shape of the route. The object in which direction the vehicle drives will be detected several times but with kind of random values. This causes the visible shape of the route in the map.
\newpage

\section{Conclusion}

The system as a whole is working reliable and fulfills almost all tasks specified at the beginning of this project. Special attention is paid to the software design principles defined in section \ref{subsec:design_principles}. While the system is running with full functionality, the average CPU load of the Raspberry Pi is considerable low. At the same time the robot reacts almost immediately to incoming events. This is mainly due to the thoughtful development of internal processes as well as outsourcing computing intensive tasks to external hardware. The robot is capable to handle further functional expansions.\\

This project was used as an opportunity to implement an economical system containing multiple platforms. The therefore necessary inter-process communication was seen as a special challenge. We decided on purpose against a supportive framework, as ROS\footnote{The Robot Operating System (ROS) is a set of software libraries and tools that help to build robot applications.} for example. On the one hand the usage of ROS introduces a rather impractical build tool (i.e. catkin), which makes it hard to use convenient techniques as remote development as well as remote debugging and restricts the usage of programming languages. On the other hand this project showed us the challenges of synchronizing multiple threads and processes on various devices. The gained experience can be used in future projects.\\

Moreover the project is well documented and fully open source. All related documents are stored and updated in a public accessible Github repository. State of the art source code documentation tools were used to create an almost consistently documentation of the source code. The created documentation are public accessible as well and linked to the Github repository.

\subsection{Known Issues}

The present system has known issues depending the hardware as well as the software, which should be addressed in future works.\\

The designed PCB board is comparatively small compared to the implemented functionality. This requires to use small circuit path sizes. Some circuit paths had an unwanted conducting connection, introduced by inaccuracies in the development process. The elimination of the unwanted connections demanded more time than planned for the PCB board. Furthermore the PCB is mounted right above the processor, which could prevent proper cooling of the Raspberry Pi. In a new design the PCB board would be designed bigger and placed below the Raspberry Pi.\\

The project revealed insufficient edge detection capabilities of the Raspberry Pi. Especially the used ultra sonic sensors are dependent on a proper edge detection. Most of the effect could be bypassed by modifications in software. However, these modifications introduce inaccuracy to the measurement result. this problem can be avoided by using a microcontroller for the sensor readings and pass the results to the Raspberry Pi.\\

Synchronization and collaboration of the threads implemented in the base process can be improved. Observation of the process log indicates rarely happening errors. This results most times in simultaneous access of shared resources. As for the time being there is no known way to reliable reproduce the error, it is a rather hard task to track and eliminate the cause. The throughout this project gained experience would be used to pay more effort to thread synchronization, in a new design of the software.\\

The ultra sonic measurements are conducted simultaneously in this project. This leads to mutual influence of the ultra sonic sensors. Therefore the ultra sonic sensors only deliver reliable results, when an object is detected in short distance (i.e below ca. 80\,cm). Mainly this prevents the mapping process from drawing an accurate map of the surroundings. A possible solution is to conduct the ultra sonic measurement one after another, although this would be against the multi-threaded principal of this project. From our point of view a different kind of distance sensor is the preferred solution.

\newpage

\section{References}
\begin{flushleft}
\begin{thebibliography}{99}
\bibitem[1]{l293dne} Texas Instruments Incorporated, ``L293x Quadruple Half - H Drivers -- Data Sheet'' 2016 [Online]. Available: \url{https://www.ti.com/lit/ds/symlink/l293d.pdf}. [Accessed: 31.01.2020].
\bibitem[2]{tcst1103} Vishay Semiconductors, ``Transmissive Optical Sensor with Phototransistor Output'' 2011 [Online]. Available: \url{https://www.vishay.com/docs/83764/tcst1103.pdf}. [Accessed: 31.01.2020].
\bibitem[3]{hc_sr04} Elec Freaks, ``Ultrasonic Ranging Module HC - SR04'' [Online]. Available: \url{https://cdn.sparkfun.com/datasheets/Sensors/Proximity/HCSR04.pdf}. [Accessed: 31.01.2020].
\bibitem[4]{bno055} Adafruit, ``BNO055 Absolute Orientation Sensor'' 2015 [Online]. Available: \url{https://learn.adafruit.com/adafruit-bno055-absolute-orientation-sensor/overview}. [Accessed: 31.01.2020].
\bibitem[5]{rpi4} Raspberry Pi foundation, ``Raspberry Pi 4 Model B'' [Online]. Available: \url{https://www.raspberrypi.org/products/raspberry-pi-4-model-b/}. [Accessed: 31.01.2020].
\end{thebibliography} 

\end{flushleft}



\end{document}